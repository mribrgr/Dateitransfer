\documentclass[a4paper, 11pt]{article}
\usepackage[ngerman]{babel}
\usepackage[utf8]{inputenc}
\pagestyle{headings}

\begin{document}

\tableofcontents

\newpage

\section{Ansatz}
\subsection{Wozu?}
Aufgaben:

sinnvolle Abstraktion der Informationen
Management

Herausforderungen:
Langsamkeit der Medien
Umfang
Fehlertoleranz
\subsection{Aufbau einer Festplatte}
Stapel von rotierenden Magnetplatten, konstante Rotationsgeschwindigkeit
Rotationsgeschwindigkeit ca. 5400 - 15000 min^-1 // Warum nicht lauter? -> zu laut
2-16 Platten
konzentrische Spuren (Tracks), ca 10.000 pro Oberfläche
übereinanderliegende Spuren = >Zylinder
kleinste ansprechbare Einheit: physischer Block (512 Byte -> kann mehr), z.B. 150-300 Sektoren pro Spur
hystorisch: Adressierung eines Sektors über \{Zylinder, Kopf, Sektor\} - Tribel
heute: Logical Block Addressing, einfach Durchnumerierung aller Blöcke (auf firmware der Festplatte)
physisches Layout vor Nutzer vorborgen: Abbildung Logischer Blocknummern auf Physische Blocknummern durch Festplattenelektronik

CD-ROM - Datenvolumen: es sollte 9. Symphonie von Beethoven drauf gehen

\subsection{Welche Dateisysteme gibt es?}
MS-DOS	FAT12, FAT16
Windows NT ... 10	NTFS
MacOS	HFS, HFS+, APFS
LInux	ext2fs, ext4fs, btrfs

Es gibt für das gleiche Betriebssystem manchmal verschiedene Dateisysteme (muss man sich vorher überlegen)

Zusätzlich gibt es BS-übergreifende Dateisysteme (z.B. für CD-ROM: ISO9660)

\subsecton{Grundlegende Abstraktionen: Datei}
Datei = "Ansammlung" von Nutzdaten + Attribute

Attribute:

Schutz: Wer darf welche Operation mit Datei ausführen
Eigentümer der Datei
Beschränkungen der erlaubten Operationen (Read-only)
Beschränkung der Sichtbarkeit der Datei(Hidden Flag, .dateiname)
Dateiname
Zeitstempel
Größe der Datei
Stellung des Dateipositionszeigers

-> stat Kommando unter Linux

\end{document}